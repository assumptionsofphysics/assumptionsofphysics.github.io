\documentclass[11pt]{article}

\usepackage[margin=1.25in]{geometry}

\usepackage{amsmath}
\usepackage{amssymb}
\usepackage{graphicx}
\usepackage{amsfonts}
\usepackage{enumitem}


\begin{document}
	

\title{Le meccaniche Newtoniana, Lagrangiana e Hamiltoniana non sono equivalenti}
\author{Assumptions of Physics Collaboration}

\date{}

\maketitle

Dopo aver concluso un corso di meccanica razionale si ha l'impressione che le meccaniche Newtoniana, Lagrangiana e Hamiltoniana siano equivalenti. Sfortunatamente, anche se questo \`e spesso quello che dicono i libri, non \`e cos\`i. Ma non credeteci sulla parola.

I libri di testo enfatizzano la trattazione di sistemi descrivibili da tutti e tre. Qui studieremo due controesempi: un corpo soggetto all'attrito dell'aria (che \`e Newtoniano ma non Lagrangiano o Hamiltoniano) ed un fotone trattato come particella classica (che \`e Hamiltonian ma non Lagrangiano o Hamiltoniano).

Guarderemo solamente la dinamica per particelle (i.e. non teorie di campo) indipendente dal tempo (i.e. dove la forza, la Lagrangiana e l'Hamiltoniana non dipendono dal tempo). In quei casi ci sono alcuni dettagli che rendono la discussione pi\`u complicata, anche se le conclusioni non cambiano di molto.

\section{Revisione veloce}

Rivediamo prima alcune definizioni chiave, cos\`i da avere un'idea precisa di quello di cui parleremo.

Un sistema \`e Newtoniano se c'\`e una massa $m$ ed una forza $F(x,v)$ per cui si pu\`o applicare la seconda legge di Newton:
\begin{equation}
\label{Fma}
F=ma
\end{equation}

Un sistema \`e Lagrangiano se ammette una funzione $L(x,v)$ tale che il sistema obbedisce le equazioni di Eulero-Lagrange:
\begin{equation}
\label{EulerLagrange}
\frac{\partial L}{\partial x} - \frac{d}{dt} \frac{\partial L}{\partial v} = 0
\end{equation}
Nota che, usando la regola della catena, abbiamo:
\begin{align*}
\frac{d}{dt} \frac{\partial L}{\partial v} &= \frac{\partial}{\partial x} \frac{\partial L}{\partial v} \frac{dx}{dt} + \frac{\partial}{\partial v}  \frac{\partial L}{\partial v} \frac{dv}{dt} \\
&= \frac{\partial^2 L}{\partial x \partial v} v + \frac{\partial^2 L}{\partial^2 v} a
\end{align*}
Possiamo riscrivere le equazioni di Eulero-Lagrange come:
\begin{equation}
\label{EulerLagrangeMod}
\frac{\partial^2 L}{\partial v^2} a + \frac{\partial^2 L}{\partial x \partial v} v - \frac{\partial L}{\partial x}=0
\end{equation}
Questo ci semplificher\`a le cose.

Un sistema \`e Hamiltoniano se esiste una funzione $H(x,p)$ tale che il sistema obbedisce le equazioni di Hamilton:
\begin{equation}
\begin{aligned}
\frac{dx}{dt} &= \frac{\partial H}{\partial p} \\
\frac{dp}{dt} &= - \frac{\partial H}{\partial x}
\end{aligned}
\label{Hamilton}
\end{equation}
Nota che, in questo caso, la velocit\`a $v=\frac{dx}{dt}$ pu\`o essere espressa come una funzione $v(x,p)$ della posizione e del momento.

In tanti casi, lo stesso sistema ammette tutte e tre le descrizioni. Una particella libera \`e uno di questi casi, visto che
\begin{equation}
\begin{aligned}
F&=0 \\
L&=\frac{1}{2}mv^2 \\
H&=\frac{1}{2}\frac{p^2}{m}
\end{aligned}
\end{equation}
danno le stesse traiettorie. La tesi \`e che ci\`o non \`e vero per i sequenti due sistemi.

\section{Particella soggetto ad attrivo viscoso}

Supponiamo di avere una particella soggetta all'attrito dell'aria. In questo caso $F=-bv$ e abbiamo:
\begin{equation}
\label{dragEq}
ma+bv=0
\end{equation}
Questo \`e compatibile con la meccanica Lgrangiana o Hamiltoniana?

Comparando \eqref{EulerLagrangeMod} con \eqref{dragEq}, potremmo provare
\begin{align}
	\label{dragMass}
	&\frac{\partial^2 L}{\partial^2 v} = m \\
	\label{dragVelocity}
	&\frac{\partial^2 L}{\partial x \partial v} v - \frac{\partial L}{\partial x} = bv
\end{align}

Integrando \eqref{dragMass}, la forma della Lagrangiana viene ristretta a:
\begin{equation*}
	L = \frac{1}{2} m v^2 + f_1(x) v + f_2 (x)
\end{equation*}
dove $f_1(x)$ e $f_2 (x)$ sono funzioni arbitrarie della posizione. Sostituendo le espression dentro \eqref{dragVelocity} abbiamo:
\begin{align*}
	\frac{\partial^2 L}{\partial x \partial v} v - \frac{\partial L}{\partial x} &= \left(\frac{\partial f_1}{\partial x}\right) v - \frac{\partial f_1}{\partial x} v -  \frac{\partial f_2}{\partial x} \\
	&= -  \frac{\partial f_2}{\partial x} \\
	&= bv
\end{align*}
che \`e impossibile visto che $f_2$ \`e solo funzione della posizione.

Per la meccanica Hamiltoniana, se assumiamo $p=mv$ abbiamo:
\begin{equation*}
	\frac{p}{m}=v= \frac{dx}{dt} = \frac{\partial H}{\partial p}
\end{equation*}
Integrando, la forma dell'Hamiltoniana deve essere
\begin{equation*}
	H=\frac{1}{2}\frac{p^2}{m} + V(x)
\end{equation*}
dove $V$ \`e una funzione arbitraria della posizione. Sostituendo nella seconda equazione:
\begin{equation*}
	\frac{dp}{dt}= ma = - \frac{\partial H}{\partial x} = - \frac{\partial V}{\partial x}
\end{equation*}
$V$ \`e funzione solo della posizione, perci\`o non possiamo ottenere una forza funzione della velocit\`a.

Unfortunately, both of these attempts are not a general proof: it just shows that a simple direct approach that mimics the free particle case does not work. In fact, we could set our Lagrangian as:
\begin{equation}\label{wrong_Lagrangian}
	L=v \log v - v + \frac{b}{m} x
\end{equation}
which gives
\begin{align*}
	\frac{\partial L}{\partial x} &= \frac{b}{m}\\
	\frac{\partial L}{\partial v} &= \log v + \frac{v}{v} - 1=\log v\\
	\frac{\partial^2 L}{\partial v^2} &= \frac{1}{v}\\
	\frac{\partial^2 L}{\partial x \partial v} &= 0.
\end{align*}
This means that \eqref{EulerLagrangeMod} becomes:
\begin{align*}
	\frac{1}{v} a + 0 \, v + \frac{b}{m} &= 0 \\
	ma + bv &= 0.
\end{align*}
It would seem we have a suitable Lagrangian, but if we look closer at \eqref{wrong_Lagrangian}, we see that the expression is defined only for non-negative velocity. Since the Lagrangian is not defined for all possible values of position and velocity, it does not describe all possible motions of the system. So the general problem is harder, more subtle and therefore more confusing!

For Hamiltonian mechanics, we could argue that drag is a dissipative force, the system does not conserve energy, and therefore it cannot be Hamiltonian since Hamiltonian systems always conserve energy. But, again, maybe there is a different expression for both conjugate momentum and energy that would give us the correct motion. So, how can we proceed?

We can study the general case if we recall Louisville's theorem. This states that volumes in phase space are conserved during Hamiltonian evolution. This means that there are no attractors. That is, it cannot have regions of phase space, like equilibrium point, toward which the motions tends to go.

Even if we assume that $p \neq mv$, me must have a state for all possible initial conditions, therefore momentum $p(x,v)$ must be a function of position and velocity. The line at zero velocity will correspond to a line in the $x/p$ plane, in phase space, and it will need to be an attractor, as the velocity will always tend to zero at infinite time. Since we said that Hamiltonian mechanics cannot have an attractor, we will never be able to find a Hamiltonian that works for a massive particle under linear drag.

We can use this result for the Lagrangian case as well. If we look at \eqref{EulerLagrangeMod}, we note that the term $\frac{\partial^2 L}{\partial v^2}$ cannot be equal to zero. If it were, the acceleration term would disappear and we would have an expression for the force. Since $\frac{\partial^2 L}{\partial v^2}$ must also be continuous, it means that it must be either always positive or always negative. This means that $\frac{\partial L}{\partial v}$ is either a strictly increasing or decreasing function, which means it is invertible.

We recognize $\frac{\partial L}{\partial v}$ as conjugate momentum, and therefore we must have a bijective map between velocity and conjugate momentum, which means the Lagrangian admits a Hamiltonian that gives equivalent equations of motions. Since we said no Hamiltonian can give us a massive particle under linear drag, this means that neither a Lagrangian can. A massive particle under drag is not a Lagrangian or a Hamiltonian system.

\section{Photon as a free particle}

This case will be much simpler. Suppose we have a photon traveling in space. It's trajectory is a line, traveled at constant speed $c$. It's momentum $p=\frac{\hbar}{\lambda}$ is conserved, where $\lambda$ is the wavelength. The system is not Newtonian: the mass is zero therefore \eqref{Fma} does not apply. Is it Hamiltonian?

The energy of the particle is $E=\hbar \nu$ where $\nu$ is the frequency. The product of the magnitude of the frequency and wavelength is euqal to the speed: $|\nu \lambda| = c$. Therefore we have:
\begin{equation*}
	E=\hbar \nu = c \frac{\hbar}{|\lambda|} = c|p| \end{equation*}
This is the energy as a function of position and momentum: it's the Hamiltonian of the system.

Let's verify that Hamilton's equations \eqref{Hamilton} give us the correct equations.
\begin{align*}
	H &= c|p| \\
	\frac{dx}{dt} &= \frac{\partial H}{\partial p} = c \frac{\partial |p|}{\partial p} = c \frac{|p|}{p} \\
	\frac{dp}{dt} &= - \frac{\partial H}{\partial x} = 0
\end{align*}
The momentum cannot be zero, since $H$ is discontinuous there. The velocity is $c$ in the direction of momentum. The momentum is conserved. The system is correctly described by Hamiltonian mechanics.

Since we have the Hamiltonian, we would be tempted to write:
\begin{equation}
	\label{Legendre}
	L = pv - H
\end{equation}
But for this to be a valid Lagrangian, we must be able to express it as a function of $x$ and $v$, and this is what we cannot do. Since the velocity is always constant and is only a function of the direction of $p$, the change from momentum to velocity loses information. Therefore while we can write $v(x,p)$, we cannot write $p(x,v)$. In other words: position and velocity are not enough to describe the state of the system. This is why we cannot write a Lagrangian for this case.

The photon treated as a particles is not a Lagrangian or Newtonian system.

\section{Conclusion}

Since we were able to find counter examples, Netwonian, Lagrangian and Hamiltonian are not equivalent: they describe different types of systems. Newtonian and Lagrangian mechanics are restricted to massive systems, the ones for which position and velocity are enough to identify the state of the system. Lagrangian and Hamiltonian mechanics are restricted to conservative systems, where no energy and no information is exchanged with the outside, where isolation holds and the system is deterministic and reversible.

``But wait!" You may say. ``This was only for the time independent case! In the time dependent case I can write a Hamiltonian that describes dissipative systems! Thus what you said it's wrong and you don't know what you are talking about." That may be true, but the time dependent case needs to be looked at with care as it is easy to get confused. It will be the topic for another time.
\end{document}
