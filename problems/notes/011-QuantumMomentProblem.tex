\documentclass[11pt]{article}

\usepackage[margin=1.25in]{geometry}

\usepackage{amsmath}
\usepackage{amssymb}
\usepackage{graphicx}
\usepackage{amsfonts}
\usepackage{enumitem}


\begin{document}


	\title{Notes on problem 11 - Quantum to classical moment problem}
	\author{Assumptions of Physics Collaboration}

	\date{}

	\maketitle

Just using expectation of $Q$ and $P$ can't work because products
are not uniquely defined.

In standard probability theory we have:
\begin{equation}
\begin{aligned}
	E[2 Q^2 P^2] &= E[\frac{1}{2}(2QP)^2] = E[\frac{1}{2}((Q+P)^2 - Q^2 -P^2)^2] \\
	&= E[(Q^2 + P^2)^2 - Q^4 - P^4]
\end{aligned}
\end{equation}
We have expressions in terms of sums and powers, which in principle we know how to convert into quantum observables. We have:
\begin{equation}
	\begin{aligned}
		E[\frac{1}{2}((Q+P)^2 - Q^2 -P^2)^2] &= tr\left[\frac{1}{2}((Q+P)^2 - Q^2 -P^2)^2 \rho\right] \\
		E[(Q^2 + P^2)^2 - Q^4 - P^4] &= tr\left[((Q^2 + P^2)^2 - Q^4 - P^4)\rho\right]
	\end{aligned}
\end{equation}
Expanding the first
\begin{equation}
	\begin{aligned}
		\frac{1}{2}tr[((Q+P)^2 - Q^2 -P^2)^2 \rho] &= \frac{1}{2}tr[(QQ +QP + PQ + PP - QQ - PP)^2 \rho] \\
		 &= \frac{1}{2}tr[(QP + PQ)^2 \rho] \\
		 &= \frac{1}{2}tr[(QPQP +QPPQ + PQQP + PQPQ) \rho] \\
	\end{aligned}
\end{equation}
Expanding the second
\begin{equation}
	\begin{aligned}
		tr[((Q^2 + P^2)^2 - Q^4 - P^4) \rho] &= tr[(QQQQ + QQPP + PPQQ + PPPP - QQQQ - PPPP) \rho] \\
		&= tr[(QQPP + PPQQ) \rho] \\
	\end{aligned}
\end{equation}
Comparing the operators, we have
\begin{equation}
	\begin{aligned}
		QPQP +QPPQ + PQQP + PQPQ &= QQPP - Q[Q,P]P + QQPP - Q[Q,P]P - QP[Q,P] \\
		&+ PPQQ - P[P, Q] Q - PQ [P, Q] + PPQQ - P [P, Q] Q \\
		&= 2 (QQPP + PPQQ) -3(-\imath\hbar)QP - 3(-\imath\hbar) PQ \\
		&= 2 (QQPP + PPQQ) +3(\imath\hbar)[Q,P]\\
		&= 2 (QQPP + PPQQ) +3(\imath\hbar)^2\\
		&= 2 (QQPP + PPQQ) -3\hbar^2\\
	\end{aligned}
\end{equation}
The two operators differ by a constant, so we do not have clear expectation for the moments on which to base the moment problem.

Follow up: suppose we get a canonical product (i.e. $Q^nP^m + P^mQ^n$), can we solve the moment problem?

Follow up: can we do a decomposition on position first, like one does for the Madelung equations?

% Useful links?
% https://link.springer.com/chapter/10.1007/978-3-319-64546-9_14#Sec3
% https://reader.elsevier.com/reader/sd/pii/S0047259X11001138?token=653E5622033C9438B6C4F5A283A5FA6C0937973767901BEB239F1528B4E1975F7808014F56512DBE27591BF593E5CE15&originRegion=us-east-1&originCreation=20220812154646
% https://www.sciencedirect.com/science/article/pii/S0047259X11001138

\end{document}
