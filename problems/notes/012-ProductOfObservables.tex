\documentclass[11pt]{article}

\usepackage[margin=1.25in]{geometry}

\usepackage{amsmath}
\usepackage{amssymb}
\usepackage{graphicx}
\usepackage{amsfonts}
\usepackage{enumitem}


\begin{document}
	
	
	\title{Notes on problem 12 - Product of observables}
	\author{Assumptions of Physics Collaboration}
	
	\date{}
	
	\maketitle
	
On closer look, things don't work.

\begin{equation}
	A \times B = \frac{\{A, B\}}{2}
\end{equation}	

The relationship is
\begin{equation}
	\begin{aligned}
	\{\{A, B\}, C\} - \{A, \{B, C\}\} = ABC+BAC-ABC-ACB = A[B,C] - [B,A]C
	\end{aligned}
\end{equation}
so the commutators are not on the same elements.

If $A$ and $C$ match, things work. Not in the other case
\begin{equation}
\begin{aligned}
	\{\{Q, P\}, Q\} - \{Q, \{P, Q\}\} &= Q[P,Q] - [P,Q]Q = Q (- \imath \hbar) - (-\imath \hbar)Q = 0 Q = 0 \\
	\{\{Q, Q\}, P\} - \{Q, \{Q, P\}\} &= Q[Q,P] - [Q,Q]P = Q (\imath \hbar) - (0) P = Q \imath \hbar. \\
\end{aligned}
\end{equation}

For products of $Q$ and $P$, the commutator transforms two variables into a constant, leading to lower order polynomial multiplied by $\hbar$. If $\hbar$ is small for the scale of the problem, then, the difference can be neglected.


\end{document}
